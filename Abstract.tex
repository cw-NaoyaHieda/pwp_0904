\documentclass[a4j,11pt]{jarticle}
\usepackage{fancybox}
\usepackage{mathptmx}
\usepackage{ascmac}
\usepackage{amsmath}
%\usepackage[top=30truemm,bottom=30truemm,left=25truemm,right=25truemm]{geometry}
\usepackage{algorithm}
\usepackage{algorithmic}
\usepackage{bm}
\usepackage[dvipdfmx]{graphicx}
\usepackage{subcaption}
\usepackage{here}
\usepackage{wrapfig}
\bibliographystyle{apalike}
%\setlength{\voffset}{-25.4mm}
\setlength{\topmargin}{-17.5mm}   %トップとヘッダの間隔
%\setlength{\headheight}{20mm}   %ヘッダの高さ
%\setlength{\headsep}{0mm}   %ヘッダとテキストの間隔
\setlength{\textwidth}{45zw}   %テキストの幅
\setlength{\hoffset}{-10mm}
\setlength{\textheight}{45\baselineskip}   %テキストの高さ
%\addtolength{\textheight}{\topskip}
%\setlength{\footskip}{0mm}
%\setlength{\oddsidemargin}{21.5mm}   %サイドとテキストの間隔(奇数ページ)
%\setlength{\evensidemargin}{21.5mm}   %サイドとテキストの間隔(偶数ページ)
\pagestyle{empty}   %ページ番号なし
\newcommand{\g}[1]{\boldsymbol{#1}}
\newcommand{\lw}[1]{\smash{\lower2.0ex\hbox{#1}}}
\renewcommand{\baselinestretch}{1.0}
\setlength\floatsep{0pt}
\setlength\textfloatsep{0pt}
\setlength\intextsep{0pt}
\setlength\abovecaptionskip{0pt}
\makeatletter
\def\theequation{\arabic{equation}}   %数式番号を(章.式)形式
\@addtoreset{equation}{section}
%\def\thefigure{\thesection.\arabic{figure}}   %図番号を(章.図)形式
%\@addtoreset{figure}{section}
%\def\thetable{\thesection.\arabic{table}}   %表番号を(章.表)形式
%\@addtoreset{table}{section}
\def\tr{\mathop{\operator@font tr}\nolimits}
\def\grad{\mathop{\operator@font grad}\nolimits}
\def\St{\mathop{\operator@font St}\nolimits}
\def\Hess{\mathop{\operator@font Hess}\nolimits}
\def\D{\mathop{\operator@font D}\nolimits}
\def\sym{\mathop{\operator@font sym}\nolimits}
\def\s.t.{\mathop{\operator@font s.t.}\nolimits}
\def\diag{\mathop{\operator@font diag}\nolimits}
\def\section{\@startsection{section}{1}{\z@}
   {0\Cvs \@plus.5\Cdp \@minus.2\Cdp}
   {0.01\Cvs \@plus.3\Cdp}
   {\normalfont \Large \bfseries}}
\makeatother
\makeatletter
\renewenvironment{thebibliography}[1]
{\section*{\refname\@mkboth{\refname}{\refname}}%
   \list{\@biblabel{\@arabic\c@enumiv}}%
        {\settowidth\labelwidth{\@biblabel{#1}}%
         \leftmargin\labelwidth
         \advance\leftmargin\labelsep
	 \setlength\itemsep{-0.7zh}%←ここの数値を調整
         \@openbib@code
         \usecounter{enumiv}%
         \let\p@enumiv\@empty
         \renewcommand\theenumiv{\@arabic\c@enumiv}}%
   \sloppy
   \clubpenalty4000
   \@clubpenalty\clubpenalty
   \widowpenalty4000%
   \sfcode`\.\@m}
  {\def\@noitemerr
    {\@latex@warning{Empty `thebibliography' environment}}%
   \endlist}

\def\JTeX{\leavevmode\lower .5ex\hbox{J}\kern-.17em\TeX}
\def\JLaTeX{\leavevmode\lower.5ex\hbox{J}\kern-.17em\LaTeX}

\makeatother
\makeatletter
  \renewcommand{\thealgorithm}
\makeatother
\makeatletter
\def\subsection{\@startsection{subsection}{1}{\z@}
   {0\Cvs \@plus.5\Cdp \@minus.2\Cdp}
   {0.01\Cvs \@plus.3\Cdp}
   {\normalfont \normalsize \bfseries}}
\makeatother
\newcommand{\figcaption}[1]{\def\@captype{figure}\caption{#1}}
\newcommand{\tblcaption}[1]{\def\@captype{table}\caption{#1}}
\makeatother
\begin{document}
\begin{center}
{\Large \textbf{粒子フィルタリングによるデフォルト率分布の推定}}\\
\begin{eqnarray*}
&&東京理科大学大学院工学研究科経営工学専攻 稗田 尚弥\\
&&東京理科大学工学部情報工学科       塩濱 敬之
\end{eqnarray*}
\end{center}
\begin{flushright}

\end{flushright}
\vspace{-3zw}
%%%%%これ以下, 本文%%%%%
\section{はじめに}
銀行の業務における適切なリスク管理手法は,経済環境の安定において必要不可欠である.
融資によって構成される与信ポートフォリオのリスクとリターンの状況を評価し,その健全性や収益性を高めていく与信ポートフォリオの管理は,銀行業務において非常に重要である.与信ポートフォリオは,融資を行った相手が債務不履行(デフォルト)するリスクを抱えている.
デフォルトをモデル化するために,企業$i$の資産価値を$U_i = a F + \sqrt{1-a^2}Z_i$で定義し,閾値$U$以下の値をとったときにその企業はデフォルトすると考える1-ファクターMertonモデルが提唱されている.ここで,$F$は全ての企業の共通ファクター,$Z_i$は各企業が個別にもつファクター,$a$は企業$i$の共通ファクターへの感応度である.本研究では連邦準備制度理事会で発表されている,米国のデフォルト率を観測変数とし,1-ファクターMartonモデルからHull (2012) と Lamb and Perraudin (2008) が求めたデフォルト率の従う分布を状態空間モデルに適用する.状態変数は,各デフォルト率分布が持つパラメータのうち,時変であると考えるべきものを状態変数とする.\\
\hspace{4mm}状態空間モデルによる推定を行う際に,非線形非ガウスな一般の状態空間モデルでは,状態の事後分布を解析的に求める事が困難である.この対応策として,多くのモデルで提案されているのが粒子フィルタリングである.本研究でも粒子フィルタリングによってパラメータ推定を行った.また,時変でないと考えたパラメータの推定は,Douc, Moulines, and Stoffer (2014) を参考にParticle MCEM アルゴリズムなど,いくつかのEMアルゴリズムを適用した.
\section{デフォルト率の状態空間モデル}
本研究では,二つの論文から二種類の状態空間モデルを考案した.下記にモデルをまとめる.\\  
1. 時変パラメータを持つVasicek 1-ファクターモデル:
\begin{eqnarray*}
&&DR_t\sim g(DR_t|PD_t,\rho_t), \\
&&g(DR_t|PD_t,\rho_t)=\sqrt{\frac{1-\rho_t}{\rho_t}}\exp{\left\{ \frac{1}{2} \left[ (N^{-1}(DR_t))^2-\left(\frac{\sqrt{1-\rho_t}N^{-1}(DR_t)-N^{-1}(PD_t)}{\sqrt{\rho_t}} \right) \right] \right\}},\\
&&PD_t\sim N(\phi_{PD}(PD_{t-1}-\mu_{PD})+\mu_{PD},\sigma_{PD}^2), \\
&&\rho_t\sim N(\phi_{\rho}(\rho_{t-1}-\mu_{\rho})+\mu_{\rho},\sigma_{\rho}^2).
\end{eqnarray*}
2. Lamb and Perraudinモデル:
\begin{eqnarray*}
&&DR_t\sim N\left( \frac{\Phi^{-1}(q)-\sqrt{\rho}\sqrt{\beta}X_{t-1}}{\sqrt{1-\rho}}, \frac{\rho(1 - \beta)}{1 - \rho}\right), \\
&&X_t\sim N\left(\sqrt{\beta}X_{t-1} , 1-\beta\right). 
\end{eqnarray*}
ここで$DR_t$は時点$t$でのデフォルト率,モデル1の$PD_t$は各企業の潜在的デフォルト率,$\rho_t$は時刻$t$における共通ファクターへの感応度,モデル2の$X_t$は各企業の共通要因を表している.また,モデル1の$\{\phi_{PD},\phi_{rho},\mu_{PD},\mu_{\rho},\sigma_{PD},\sigma_{\rho}\}$,モデル2の$\{q,\rho,\beta\}$が,時変でないと考えたパラメータである.
\makeatletter
\def\@biblabel#1{}
\makeatother
\begin{thebibliography}{n}
\bibitem{key3}  J. C. Hull, Risk Management and Financial Institution, 2012, New York: Wiley.
\bibitem{key4}  R. Lamb and W. Perraudin, Dynamic Default Rates, 2008, Working Paper.
\bibitem{key16} R. Douc and E. Moulines and D.S Stoffer, Nonlinear time series. Theory, methods and applications with R examples, 2014, CRC Press.
\end{thebibliography}
\end{document}